%%%%%%%%%%%%%%%%%%%%%%%%%%%%%%%%%%%%%%%%%%%%
% En 'inclues.tex' se encuentran la importación de paquetes necesarios
%%%%%%%%%%%%%%%%%%%%%%%%%%%%%%%%%%%%%%%%%%%%
\input{project_settings}
\input{listings_settings}


\begin{document}

%%%%%%%%%%%%%%%%%%%%%%%%%%%%%%%%%%%%%%%%%%%%
% En 'titlepage.tex' se encuentra la página de título
%%%%%%%%%%%%%%%%%%%%%%%%%%%%%%%%%%%%%%%%%%%%
\input{titlepage}

%%%%%%%%%%%%%%%%%%%%%%%%%%%%%%%%%%%%%%%%%%%%
% INDICE
%%%%%%%%%%%%%%%%%%%%%%%%%%%%%%%%%%%%%%%%%%%%
\clearpage
\tableofcontents
\clearpage 

\lstset{style=bashstyle}

\section{\emph{Footprinting}}

El término \emph{Footprinting} se refiere al proceso de recolectar la mayor cantidad de información posible sobre un sistema objetivo con el fin de encontrar formas de penetrarlo \autocite{FootprintingOpenCampus}. Este etapa previa a realizar un ataque, conocida como \emph{fase de reconocimiento}, el atacante intenta encontrar información como \autocite{FootprintingCiberinformatico}:

\begin{itemize}
    \item Rango de Red y sub-red (\emph{Network Range} y \emph{subnet mask})
    \item Acertar máquinas o computadoras activas
    \item Puertos abiertos y las aplicaciones que están corriendo en ellos
    \item Detectar versiones de sistemas operativos
    \item Nombres de Dominios (\emph{Domain Names})
    \item Bloques de Red (\emph{Network Blocks})
    \item Direcciones IP específicas
    \item País y ciudad donde se encuentran los servidores
    \item Información de contacto (números telefónicos, emails)
    \item \emph{DNS records}
\end{itemize}

Mucha de la información antes mencionada, como Domain Names, algunas direcciones IP, país, ciudad, e información de contacto puede ser conseguida consultando a las bases de datos de \emph{whois}. Esto se realiza justamente con el comando \texttt{whois} y el nombre del \emph{dominio} al cual se quiere consultar. Por ejemplo, si se desea conocer información sobre el dominio \emph{facebook.com}, se debe realizar la siguiente invocación a \texttt{whois}:

\begin{lstlisting}
    whois facebook.com
\end{lstlisting}

Además del comando \emph{whois}, que recupera información detallada sobre el dominio consultado (quién es su dueño, fecha de registro, fecha de expiración, entre otros), otras herramientas para hacer consultas a DNS son los comandos \texttt{nslookup} y \texttt{dig}. Para hacer \emph{enumeración de DNS} (obtener todos los subdominios registrados bajo un dominio) existen herramientas como \texttt{fierce}, \texttt{dnsrecon}, o \texttt{dnsenum} \autocite{DNSEnumerationTools}.   

\emph{Elija dos organizaciones cualesquiera y utilizando WHOIS y DIG, averigüe toda la información que pueda: servidores de correo, servidores DNS, Servidores WEB, etc} 
~\\

Dentro del directorio \dq{assets} se incluye un \emph{script} nombrado \dq{footprinting.sh}, el cual recibe un nombre de dominio y realiza varias consultas con los comando \texttt{dig} y \texttt{whois}. A continuación se incluye un ejemplo de ejecución con el dominio \emph{github.com} y las partes más importantes de su salida

\begin{lstlisting}
   bash footprinting.sh github.com 

   >>>

    *** dig -t NS +short github.com ***
    ns3.p16.dynect.net.
    ns1.p16.dynect.net.
    ns4.p16.dynect.net.
    ns-520.awsdns-01.net.
    ns-1283.awsdns-32.org.
    ns2.p16.dynect.net.
    ns-1707.awsdns-21.co.uk.
    ns-421.awsdns-52.com.

    *** dig -t MX +short github.com ***
    1 ASPMX.L.GOOGLE.com.
    5 ALT1.ASPMX.L.GOOGLE.com.
    5 ALT2.ASPMX.L.GOOGLE.com.
    10 ALT3.ASPMX.L.GOOGLE.com.
    10 ALT4.ASPMX.L.GOOGLE.com.

    *** dig -t SOA +short github.com ***
    ns1.p16.dynect.net. hostmaster.github.com. 1538412644 3600 600 604800 60

    *** whois github.com ***
       Domain Name: GITHUB.COM
       Registry Domain ID: 1264983250_DOMAIN_COM-VRSN
       Registrar WHOIS Server: whois.markmonitor.com
       Registrar URL: http://www.markmonitor.com
       Updated Date: 2017-06-26T16:02:39Z
       Creation Date: 2007-10-09T18:20:50Z
       Registry Expiry Date: 2020-10-09T18:20:50Z
       .
       .
       .
\end{lstlisting}

\emph{Visite el sitio http://www.netcraft.net/ y pruebe la funcionalidad del mismo contra el dominio www.unp.edu.ar} 
~\\

Algunos de los datos que indica sitio \emph{www.netcraft.com} sobre el dominio de la UNP son los siguientes:

\begin{itemize}
    \item Título del sitio: \emph{Universidad Nacional de la Patagonia San Juan Bosco} 
    \item Visto por primera vez en \emph{Junio de 1998} 
    \item Lenguaje primario \emph{español} 
    \item Puntaje de 7 sobre 10 en \emph{Netcraft Risk Rating} \footnote{Aunque algunos sitios tengan contenido no malicioso, \emph{Netcraft Extension} puede asignar un valor alto de riesgo porque está siendo servido bajo un dominio recientemente agregado a la base de datos de \emph{Netcraft}, porque el sitio nunca fue visto en la \emph{Netcraft Web Server Survey}, o porque la red que sirve el sitio ha servido sitios fraudulentos en el pasado. Distintos factores son tomados en cuenta \autocite{NetcraftRiskRatingFAQ}} 
    \item Dominio \emph{unp.edu.ar} 
    \item Dirección IPv4 \emph{170.210.88.21} 
    \item \emph{Nameserver} \emph{chenque.unp.edu.ar}  
    \item Administrador de DNS \emph{hostmaster@unp.edu.ar} 
\end{itemize}

\emph{Visite el sitio http://www.archive.org/web/web.php y pruebe la funcionalidad del mismo contra el sitio web de la UNP: www.unp.edu.ar. ¿Qué ventajas presenta esta herramienta respecto de otras herramientas de footprinting?} 
~\\

A diferencia de herramientas como \texttt{dig} y \texttt{whois}, el sitio \emph{www.archive.org} se dedica a visitar sitios web y tomarles un \emph{snapshot} de su estado actual. Al hacerle una consulta sobre algún sitio en particular, muestra los distintos cambios por los cuales ha pasado, pudiendo ver versiones anteriores. También brinda herramientas para visualizar la cantidad de archivos tipo MIME con los cuales ha contado el sitio (ya sean imágenes, hojas de estilo, archivos con código Javascript, y demás). Por último, recolecta y muestra las distintas \emph{URLs} que publica el sitio, con los recursos a los cuales se pueden acceder a través de la \emph{URL}.

\section{\emph{Fingerprinting} de servidores}

\begin{itemize}
    \item El sitio \emph{www.google.com} utiliza como servidor web \emph{Google Web Server} (\emph{GWS}), pero no se puede saber por las cabeceras HTTP de la respuesta qué versión de servidor usa
    \item El sitio \emph{www.ing.unp.edu.ar} indica que está usando la versión 1.10.3 del servidor web \emph{NginX}  
    \item El sitio \emph{www.microsoft.com} utiliza como servidor web \emph{Apache}, pero no se puede saber por las cabeceras HTTP de la respuesta qué versión de servidor usa
    \item El sitio \emph{serconex.juschubut.gov.ar} utiliza un servidor web \emph{Microsoft-IIFS}, versión 10.0
\end{itemize}

\section{Escaneo}

El \emph{escaneo} ó \emph{scanning} es una actividad que consiste en detectar distintos dispositivos conectados a la red, pudiendo saber qué sistema operativo están corriendo, qué puertos tienen abiertos, o qué servidores están atendiendo peticiones. 

\begin{itemize}
    \item Escaneo de hosts: se puede realizar con \texttt{nmap} o con un pequeño script en \emph{bash} que envíe un paquete utilizado \texttt{ping} a cada IP posible de la red (sabiendo la dirección de la red y su máscara, se puede calcular cuántas IPs habrán). De esta forma se puede averiguar cuántos dispositivos hay conectados a la red (aunque \texttt{nmap} brinda más información)
    \item Escaneo de puertos: también es posible hacerlo con \texttt{nmap}, indicándole una IP en particular (o combinándolo con \emph{bash}, para que itere sobre varias dirección IP). De esta forma se puede saber qué puertos tiene abierto un \emph{host} y qué servicios ofrece (también se podría intentar explotar alguna posible vulnerabilidad)
    \item Escaneo de redes WiFi: las placas modernas de red lo hacen automáticamente. Detectan qué redes WiFi hay en su rango de alcance, indicando su \emph{SSID} (si es que es público), si requieren contraseña, con qué algoritmo de encriptación trabajan para manejar las contraseñas, entre otros
    \item Escaneo de dispositivos bluetooth: se podría hacer con un dispositivo Android. De esta forma se puede detectar qué otros dispositivos están a la escucha de conexiones bluetooth, y se podría de estar forma intentar explotar alguna vulnerabilidad de la versión de bluetooth que estén corriendo.
\end{itemize}

\emph{Indique qué tipo de escaneo (hosts, puertos, vulnerabilidades, WiFi) es posible realizar}

\begin{itemize}
    \item \emph{Sólo manipulando el protocolo ARP:} \emph{hosts} (misma red)
    \item \emph{Sólo manipulando el protocolo ICMP:} \emph{hosts} (no es necesario estar en la misma red)
    \item \emph{Sólo manipulando el protocolo TCP:} puertos (no es necesario estar en la misma red).
    \item \emph{Sólo manipulando el protocolo UDP:} puertos (no es necesario estar en la misma red).
    \item \emph{Interpretando en forma pasiva tráfico de red (LAN o algún tipo de radiofrecuencias):} WiFi (misma red)
\end{itemize}

\section{Escaneo de puertos}

Para verificar los puertos abiertos en la máquina virtual de Kali, primero se ejecutó el comando \texttt{netstat} con dos combinaciones de parámetros distintas 

\begin{lstlisting}[title={Ningún puerto abierto}]
root@kali:~# netstat -nat
Active Internet connections (servers and established)
Proto Recv-Q Send-Q Local Address           Foreign Address         State
\end{lstlisting}

\begin{lstlisting}[title={Ningún puerto abierto}]
root@kali:~# netstat -nltp4
Active Internet connections (only servers)
Proto Recv-Q Send-Q Local Address           Foreign Address         State       PID/Program name
\end{lstlisting}

Luego, la salida que se obtiene de ejecutar \texttt{nmap} es consistente con lo indicado por \texttt{netstat}  

\begin{lstlisting}
root@kali:~# nmap 127.0.0.1

Starting Nmap 7.40 ( https://nmap.org ) at 2018-10-05 06:58 EDT
Nmap scan report for localhost (127.0.0.1)
Host is up (0.0000040s latency).
All 1000 scanned ports on localhost (127.0.0.1) are closed

Nmap done: 1 IP address (1 host up) scanned in 0.15 seconds
\end{lstlisting}

\begin{lstlisting}[title={Escaneo de los 65535 puertos disponibles}]
root@kali:~# nmap -p- 127.0.0.1

Starting Nmap 7.40 ( https://nmap.org ) at 2018-10-05 06:58 EDT
Nmap scan report for localhost (127.0.0.1)
Host is up (0.0000020s latency).
All 65535 scanned ports on localhost (127.0.0.1) are closed

Nmap done: 1 IP address (1 host up) scanned in 0.45 seconds
\end{lstlisting}

Ahora bien, se abrimos un puerto TCP con el comando \texttt{ncat}, la salida tanto de \texttt{netstat} como de \texttt{nmap} cambia acordemente   

\begin{lstlisting}[title={Ponemos a escuchar al puerto 8080 por conexiones TCP}]
ncat -l 8080 
\end{lstlisting}

\begin{lstlisting}[title={\texttt{netstat} indica un puerto abierto}]
root@kali:~# netstat -nltp4
Active Internet connections (only servers)
Proto Recv-Q Send-Q Local Address           Foreign Address         State       PID/Program name    
tcp        0      0 0.0.0.0:8080            0.0.0.0:*               LISTEN      2354/ncat
\end{lstlisting}

\begin{lstlisting}[title={\texttt{nmap} indica 65534 puertos cerrados, y uno abierto}]
root@kali:~# nmap -p- 127.0.0.1

Starting Nmap 7.40 ( https://nmap.org ) at 2018-10-05 07:14 EDT
Nmap scan report for localhost (127.0.0.1)
Host is up (0.0000020s latency).
Not shown: 65534 closed ports
PORT     STATE SERVICE
8080/tcp open  http-proxy

Nmap done: 1 IP address (1 host up) scanned in 0.58 seconds
\end{lstlisting}


\subsection{Pruebas con \texttt{hping3}}

\emph{Escaneo del puerto TCP/80 de la máquina local (localhost)} 

\begin{itemize}
    \item Comando a ejecutar: \texttt{hping3 -c 3 -p 80 -S localhost} 
    \begin{itemize}
        \item En la salida de \texttt{tcpdump} se puede ver que se está mandando un paquete TCP con la bandera \texttt{SYN} al puerto 80 (\emph{Half-open SYN flag scanning})
        \item La respuesta a dicho envío es un paquete con las banderas \texttt{RST} y \texttt{ACK}; a partir de esa respuesta se puede deducir que \textbf{el puerto estaba cerrado}   
    \end{itemize}
\end{itemize}

\emph{Escaneo del puerto TCP/113 de la máquina local (localhost)} 

\begin{itemize}
    \item Comando a ejecutar: \texttt{hping3 -c 3 -p 113 -S localhost} 
    \begin{itemize}
        \item En la salida de \texttt{tcpdump} se puede ver que se está mandando un paquete TCP con la bandera \texttt{SYN} al puerto 113 (\emph{Half-open SYN flag scanning})
        \item La respuesta a dicho envío es un paquete con las banderas \texttt{RST} y \texttt{ACK}; a partir de esa respuesta se puede deducir que \textbf{el puerto estaba cerrado}   
    \end{itemize}
\end{itemize}

\emph{Escaneo del puerto UDP/631 de la máquina local (localhost)} 

\begin{itemize}
    \item Comando a ejecutar: \texttt{hping3 -c 3 -p 631 -2 localhost} 
    \begin{itemize}
        \item En la salida de \texttt{tcpdump} se puede ver que se está mandando un paquete UDP al puerto 631 (\emph{UDP ICMP port unreachable scanning})
        \item La respuesta a dicho envío es un mensaje ICMP indicando que \textbf{el puerto UDP 631 está inalcanzable} 
    \end{itemize}
\end{itemize}

\emph{Escaneo del puerto UDP/53 de la máquina local (localhost)} 

\begin{itemize}
    \item Comando a ejecutar: \texttt{hping3 -c 3 -p 53 -2 localhost} 
    \begin{itemize}
        \item En la salida de \texttt{tcpdump} se puede ver que se está mandando un paquete UDP al puerto 53 (DNS) (\emph{UDP ICMP port unreachable scanning})
        \item La respuesta a dicho envío es un mensaje ICMP indicando que \textbf{el puerto UDP 53 está inalcanzable} 
    \end{itemize}
\end{itemize}

\subsection{\emph{IDLE SCAN}}

\emph{¿Qué características debe reunir un host que se pueda utilizar como zombie?} 

Para que un \emph{host} sea utilizado como \emph{zombie}, su tráfico en la red debe ser mínimo, de modo que el ID de los paquetes IP (\emph{IPID}) sólo incremente con los paquetes enviados para el \emph{IDLE SCAN} (o que el incremento sea predecible).

\subsection{Detección de escaneo de puertos}

Una forma de detectar y evitar escaneo de puertos, sería bloquear las direcciones IPs de aquellos \emph{hosts} que estén realizando alguna actividad sospechosa como por ejemplo enviar un paquete TCP con las banderas \texttt{SYC+ACK} cuando no se estaban esperando.

Con respecto a la detección de un \emph{idle scan}, se debería evitar el uso de valores de ID de los paquetes IP enviados predecibles.  

Cabe aclarar que existen implementaciones tanto de software como de hardware (IDS) para detectar escaneo de puertos, de hosts, de vulnerabilidades, entre otros.


\subsection{\emph{Fingerprinting} de sistemas operativos}

En \autocite{OSFingerprinting} se da la siguiente definición de \emph{fingerprinting de sistemas operativos}:

\begin{quote}
    (\emph{OS Fingerprinting}) es el proceso de recopilación de información que permite identificar el sistema operativo en el ordenador que se tiene por objetivo. El \textbf{OS Fingerprinting activo} se basa en el hecho de que cada sistema operativo responde de forma diferente a una gran variedad de paquetes malformados. De esta manera, utilizando herramientas que permitan comparar las respuestas con una base de datos con referencias conocidas, es posible identificar cuál es el sistema operativo.
    
    [...] El \textbf{OS Fingerprinting pasivo} no se realiza directamente sobre el sistema operativo objetivo. Este método consiste en el análisis de los paquetes que envía el propio sistema objetivo a través de técnicas de sniffing. De esta forma, es posible comparar esos paquetes con una base de datos donde se tenga referencias de los distintos paquetes de los diferentes sistemas operativos y, por lo tanto, es posible identificarlos.
\end{quote}

La finalidad de llevar a cabo un proceso de \emph{fingerprinting de sistemas operativos} consiste en \textbf{determinar qué sistema operativo está corriendo} cierto \emph{host} dentro de una red.   

El comando \texttt{nmap} permite hacer \emph{fingerprinting de sistemas operativos}. Para ello, se lo debe ejecutar con la bandera \texttt{-o}   

\begin{lstlisting}[title={Se debe ejecutar el comando con privilegios de superusuario}]
$ sudo nmap -O <IP>
\end{lstlisting}

\begin{lstlisting}[title={\emph{OS Fingerprinting} a un Ubuntu 16.04}]
$ sudo nmap -O 192.168.0.16

Starting Nmap 7.70 ( https://nmap.org ) at 2018-10-05 11:44 -03
Nmap scan report for 192.168.0.16
Host is up (0.0054s latency).
Not shown: 999 closed ports
PORT     STATE SERVICE
8080/tcp open  http-proxy
MAC Address: 30:3A:64:2C:0E:25 (Intel Corporate)
Device type: general purpose
Running: Linux 3.X|4.X
OS CPE: cpe:/o:linux:linux_kernel:3 cpe:/o:linux:linux_kernel:4
OS details: Linux 3.2 - 4.9
Network Distance: 1 hop

OS detection performed. Please report any incorrect results at https://nmap.org/submit/ .
Nmap done: 1 IP address (1 host up) scanned in 4.61 seconds
[luciano@arch-bangho:~/Documents] bash $ sudo nmap -o 192.168.0.16
Starting Nmap 7.70 ( https://nmap.org ) at 2018-10-05 11:45 -03
WARNING: No targets were specified, so 0 hosts scanned.
Nmap done: 0 IP addresses (0 hosts up) scanned in 0.03 seconds
\end{lstlisting}

\begin{lstlisting}[title={\emph{OS Fingerprinting} a un Debian 9}]
$ sudo nmap -O 192.168.0.48

Starting Nmap 7.70 ( https://nmap.org ) at 2018-10-05 11:47 -03
Nmap scan report for pcdebian1 (192.168.0.48)
Host is up (0.0045s latency).
Not shown: 999 closed ports
PORT   STATE SERVICE
22/tcp open  ssh
MAC Address: 84:16:F9:11:DA:02 (Tp-link Technologies)
Device type: general purpose
Running: Linux 3.X|4.X
OS CPE: cpe:/o:linux:linux_kernel:3 cpe:/o:linux:linux_kernel:4
OS details: Linux 3.2 - 4.9
Network Distance: 1 hop

OS detection performed. Please report any incorrect results at https://nmap.org/submit/ .
Nmap done: 1 IP address (1 host up) scanned in 2.43 seconds
\end{lstlisting}


\subsection{\emph{Fingerprinting} de servicios}

\emph{¿Cuál es la finalidad de realizar fingerprinting de servicios? ¿banner grabbing es la forma más sencilla de realizarlo?} 

La finalidad del \emph{fingerprinting de servicios} es \textbf{qué servicios y qué versiones de los servicios tiene corriendo un \emph{host} en un puerto bien conocido}. Una de las formas más sencillas de realizarlo es a través del \emph{banner grabbing}, que consiste en determinar el servicio y su versión en base a la respuesta inicial que dio el servidor al mensaje enviado al puerto bien conocido.

Un ejemplo de esto sería enviarle un paquete al puerto 22 de un servidor, y verificar qué servidor de SSH está atendiendo.

\section{Enumeración}

\emph{¿Qué es enumeración?} 
~\\

La enumeración tiene lugar después de la etapa de \emph{escaneo} (\emph{footprinting}, \emph{fingerprinting}, \emph{Google hacking}) y es el proceso de recopilación y compilación de nombres de usuario, nombres de las máquinas, recursos de red, recursos compartidos y servicios que se pueden conseguir de un objetivo \autocite{EhackEnumeracion}.  

\subsection{Enumeración de distintos protocolos}

\begin{itemize}
    \item Redes \emph{WiFi} 
    \begin{itemize}
        \item \texttt{nmcli dev wifi} 
        \item \texttt{sudo iw dev $<$INTERFAZ\_INALÁMBRICA$>$ scan} 
    \end{itemize}
    \item Dispositivos \emph{bluetooth} (la computadora que realiza el escaneo debe contar con un dispositivo bluetooth y estar encendido) 
    \begin{itemize}
        \item \texttt{hcitool scan} 
    \end{itemize}
    \item Recursos presentes de un red Windows
    \begin{itemize}
        \item \texttt{nbtscan $<$DIRECCION\_RED$>$/$<$MÁSCARA$>$} 
    \end{itemize}
    \item Información de DNS de algún dominio en particular
    \begin{itemize}
        \item \texttt{dig any +nocmd +multiline +noall +answer $<$DOMINIO$>$} 
    \end{itemize}
\end{itemize}

\subsection{Enumeración de DNS con Kali Linux}

Dentro del directorio \dq{assets} se encuentra el archivo \dq{dnsenum-unpeduar.txt} a modo de evidencia del experimento de enumeración de DNS con \texttt{dnsenum}. 








%%%%%%%%%%%%%%%%%%%%%%%%%%%%%%%%%%%%%%%%%%%%
% FIN DOCUMENTO, AHORA REFERENCIAS
%%%%%%%%%%%%%%%%%%%%%%%%%%%%%%%%%%%%%%%%%%%%
\clearpage
\printbibliography

\end{document}
