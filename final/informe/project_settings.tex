%%%%%%%%%%%%%%%%%%%%%%%%%%%%%%%%%%%%%%%%%
% University Assignment Title Page 
% LaTeX Template
% Version 1.0 (27/12/12)
%
% This template has been downloaded from:
% http://www.LaTeXTemplates.com
%
% Original author:
% WikiBooks (http://en.wikibooks.org/wiki/LaTeX/Title_Creation)
%
% License: CC BY-NC-SA 3.0 (http://creativecommons.org/licenses/by-nc-sa/3.0/)
% 
% Instructions for using this template:
% This title page is capable of being compiled as is. This is not useful for 
% including it in another document. To do this, you have two options: 
%
% 1) Copy/paste everything between \begin{document} and \end{document} 
% starting at \begin{titlepage} and paste this into another LaTeX file where you 
% want your title page.
% OR
% 2) Remove everything outside the \begin{titlepage} and \end{titlepage} and 
% move this file to the same directory as the LaTeX file you wish to add it to. 
% Then add \input{./title_page_1.tex} to your LaTeX file where you want your
% title page.
%
%%%%%%%%%%%%%%%%%%%%%%%%%%%%%%%%%%%%%%%%%
%\title{Title page with logo}
%----------------------------------------------------------------------------------------
%	PACKAGES AND OTHER DOCUMENT CONFIGURATIONS
%----------------------------------------------------------------------------------------
\PassOptionsToPackage{warn}{textcomp}
\documentclass[14pt]{extarticle}
%Paquetes para idioma español y codifcación UTF8
\usepackage[spanish]{babel}
\usepackage[utf8]{inputenc}
\usepackage{csquotes}

%%% BIBLATEX
\usepackage{biblatex}
%%% BIBLIOGRAPHY
\addbibresource{references.bib}

%fuente 'fourier'
\usepackage{fourier}
%paquete para URLs
\usepackage{url}
\usepackage[hidelinks]{hyperref}
%paquete para ubicar las imágenes
\usepackage{float}
%paquete para imágenes y en dónde las tiene que buscar
\usepackage{graphicx}
\graphicspath{{images/}}
%paquete para epígrafes
\usepackage{subcaption}
%paquete para definir los márgenes de la hoja
\usepackage[left=1.5cm,right=1.5cm,top=3cm,bottom=3cm]{geometry}
%paquete para poner todos y comentarios
\usepackage[colorinlistoftodos]{todonotes}
%paquete para trabajar con código
\usepackage{listings}
%paquete para trabajar con colores y definir propios
\usepackage{color}

%paquete para el checkmark y la cruz
\usepackage{pifont}
%paquete para el signo de copyright
\usepackage{textcomp}

%paquete para que los \texttt{} no rompan el margen de la página
\usepackage[htt]{hyphenat}

\usepackage{enumerate}
%paquete para armar layouts multicolumna
\usepackage{multicol}


%Cabeceras
\usepackage{fancyhdr}
\pagestyle{fancy}
\fancyhead[L]{Administración de Redes y Seguridad, 2018}
\fancyhead[C]{}
\fancyhead[R]{UNPSJB}

\fancyfoot[R]{Luciano Serruya Aloisi}

%Comando para poner doble comillas más fácil
\newcommand{\dq}[1]{``#1''}
\newcommand{\cmark}{\ding{51}}
\newcommand{\xmark}{\ding{55}}
